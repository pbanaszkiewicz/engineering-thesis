\chapter{Specyfikacja projektu}
\label{cha:specyfikacja_projektu}

W tym rozdziale opisane zostało wykonanie projektu zgodnie z wymaganiami postawionymi w rozdziale \ref{cha:wstep}.

%---------------------------------------------------------------------------

\section{Opis części sprzętowej}
\label{sec:opis_cz_sprzetowej}

W poniższym rozdziale została opisana platforma sprzętowa modułów A i B oraz ich współdziałanie z innymi komponentami, m.in. z zasilaniem akumulatorowym czy z przekaźnikiem w samochodzie elektrycznym.

\subsection{Moduł A}
\label{subsec:modul_a}

Jest to moduł bazowy, tj. trzymany przez operatora w dłoni. Ze względu na użycie Arduino Pro 328 5V 16MHz \cite{Ard00} (zob. zdjęcie \ref{fig:arduino_pro}) jako platformy sprzętowej, ma on niewielkie wymiary.

\begin{figure}[h]
	\centering
	\includegraphics[scale=0.3]{pics/arduino_pro_scaled.jpg}
	\caption{\label{fig:arduino_pro}Platforma sprzętowa Arduino Pro 328 5V 16MHz (zdjęcie: Piotr Banaszkiewicz).}
\end{figure}

\subsection{Moduł B}
\label{subsec:modul_b}

Jest to moduł, który został zamontowany w samochodzie. Jako platformę użyte zostało ponownie Arduino Pro 328 5V 16MHz \cite{Ard00} (zob. zdjęcie \ref{fig:arduino_pro}). Podstawowa funkcja, tj. odłączanie i załączanie silnika elektrycznego, odbywa się za pomocą przekaźnika ASR-90DD-H firmy Anly Electronics (zob. zdjęcie \ref{fig:SSR_relay} i rozdział \ref{subsec:odlaczanie_zalaczanie_silnika}).  %% DODAJ TO!!!

\begin{figure}[h]
	\centering
	\includegraphics[scale=0.3]{pics/Anly_ASR90DDH.jpg}
	\caption{\label{fig:SSR_relay}Przekaźnik półprzewodnikowy ASR-90DD-H  (zdjęcie: Piotr Banaszkiewicz).}
\end{figure}

\subsection{Zasilanie}
\label{subsec:zasilanie}

Obydwa moduły sprzętowe są zasilane z powerbanków Solo5 10000 mAh firmy Romoss. Posiadają one wbudowane różne zabezpieczenia, m.in.: natężeniowe (dla zachowania żywotności), temperaturowe czy zwarciowe (ang. \textit{short-circuit}). Mają również układ, który wyłącza powerbank w sytuacji, gdy prąd ładowania jest bardzo niewielki.  %% DODAJ BIBLIO?

Moduły A i B pobierają bardzo niewiele prądu (zmierzone: między 20 a 50 mA), więc aby powstrzymać powerbanki przed rozłączaniem zwarto masę i źródło energii dwunastoma (w przypadku modułu A) lub trzynastoma (w przypadku modułu B) rezystorami $ 910 \Omega $ wpiętymi równolegle.

Połączenie równoległe takiej ilości rezystorów daje dość niską rezystancję zastępczą (teoretycznie: $ 75.83 \Omega $ dla modułu A, $ 70 \Omega $ dla modułu B; praktycznie: występują różnice w faktycznych wartościach rezystancji rezystorów, zwłaszcza gdy te się nagrzeją) i pozwala na obniżenie wydzielanego ciepła na rezystorach, co skutkuje:

\begin{itemize}
\item większym poborem prądu, dzięki czemu powerbanki się nie wyłączają,
\item nagrzewaniem rezystorów (z prawa Ohma: $ P = \dfrac{U^2}{R} = \dfrac{(5.1)^2}{900} = 0.0289 [W] $, gdzie $ 5.1 [V] $ to zmierzone napięcie na rezystorze, a $ 900 [\Omega] $ to przybliżona rezystancja nagrzanego rezystora), a więc w przypadku 12 rezystorów położonych bardzo blisko siebie mamy około $ 0.34 [W] $ wydzielanej mocy w tym miejscu.
\end{itemize}

Ponadto banki energii umożliwiają jednoczesne ładowanie jak i rozładowywanie, tzn. jest możliwe wpięcie źródła zasilania na wejście powerbanku, jak i obciążenia na jego wyjście. Dzięki temu powerbank zasilający moduł B można ładować z portu USB znajdującego się w~samochodzie.

Sama pojemność 10000mAh teoretycznie\footnote{W praktyce pojemność 10000mAh jest nieosiągalna, gdyż powerbanki posiadają mechanizmy wydłużające ich żywotność; polega to na ładowaniu akumulatorów powerbanku tylko do pewnego stopnia.} pozwala na zasilanie układu pobierającego 100mA przez 100 godzin lub układu pobierającego 200mA przez 50 godzin. Każdy z modułów A i B po obciążeniu przez rezystory pobierał między 100 a 200 mA. Poziom naładowania powerbanków można obserwować na ich diodowych wskaźnikach.

\subsection{Komunikacja}
\label{subsec:komunikacja}

Do komunikacji wykorzystano parę modułów radiowych HC-12 433 MHz \cite{HC12}, które łączą się z~platformami sprzętowymi za pomocą łącza szeregowego. Prędkość komunikacji takiego modułu w powietrzu wynosi 5000bps, czyli typowe pakiety przesyłane przez system (zob. rozdział \ref{subsec:format_pakietow}), mające rozmiar $ 4B=32b $ są przesyłane w ułamku sekundy.

\subsection{Obudowy i montaż}
\label{subsec:obudowy_montaz}

Dla każdego modułu została zaprojektowana obudowa. Na zdjęciu \ref{fig:arduino_pro} widać cztery otwory w platformie Arduino (dwa przy lewej krawędzi płytki PCB oraz dwa przy prawej krawędzi), które zostały użyte do przykręcenia każdej z platform do obudowy.

Ponadto na każdą platformę Arduino przygotowano nakładaną na piny płytkę uniwersalną z~przylutowanymi komponentami wykorzystywanymi przez dany moduł.  %% DODAJ ZDJĘCIE PŁYTEK UNIWERSALNYCH

Obudowy zostały wydrukowane w drukarce 3D.  %% DODAJ ZDJĘCIE PROJEKTÓW OBUDÓW

\subsection{Schematy połączeń}
\label{subsec:schematy_polaczen}

Symboliczny sposób połączenia modułu A, wejścia zasilającego USB oraz transceivera radiowego został przedstawiony na schemacie \ref{fig:symbolic_schema_A}.

\begin{figure}[H]
	\centering
	\includegraphics[scale=0.4]{schemas/schema_moduleA_schem.png}
	\caption{\label{fig:symbolic_schema_A}Schemat połączeń modułu A wygenerowany w programie Fritzing \cite{Fritzing} (opracowanie własne).}
\end{figure}

Bardzo podobnie wygląda symboliczny schemat połączeń modułu B, z tym że on dodatkowo steruje przekaźnikiem (schemat \ref{fig:symbolic_schema_B}).

\begin{figure}[H]
	\centering
	\includegraphics[scale=0.4]{schemas/schema_moduleB_schem.png}
	\caption{\label{fig:symbolic_schema_B}Schemat połączeń modułu B wygenerowany w programie Fritzing \cite{Fritzing} (opracowanie własne).}
\end{figure}

\subsection{Odłączanie i załączanie silnika samochodu}
\label{subsec:odlaczanie_zalaczanie_silnika}

Pomiędzy kabel zasilający a moduł sterujący pracą silnika samochodu został wpięty przekaźnik ASR-90DD-H (zob. zdjęcie \ref{fig:SSR_relay}). Moduł sterujący odpowiedzialny jest za przetworzenie napięcia stałego na prąd zmienny poprowadzony bezpośrednio na uzwojenia silnika.

Przekaźnik ASR-90DD-H jest to przekaźnik półprzewodnikowy sterowany napięciem 3–32VDC, a zatem typu \textit{Normally Open}. Przekaźnik jest sterowany wyjściem z modułu B (zob. element ,,RELAY'' na schemacie \ref{fig:symbolic_schema_B}).

Według danych producenta, przekaźnik jest w stanie obsłużyć na wyjściu mocy do 90A prądu stałego pod napięciem do 240V. Wybrano przekaźnik mocno przekraczający spodziewane natężenie prądu, gdyż zapewnia to mniejsze wydzielanie ciepła (zob. rozdział \ref{subsec:chlodzenie_SSR}).

Niestety producent nie udostępnia schematu elektrycznego przekaźnika w dokumentacji \cite{ASR90DDH}, więc podczas jego montażu zadbano o kilka elementów bezpieczeństwa:

\begin{itemize}
\item zastosowano podkładkę silikonową jako izolator elektryczny,
\item zastosowano radiator chłodzący (widoczny na zdjęciu \ref{fig:radiator}).
\end{itemize}

\begin{figure}[h]
	\centering
	\includegraphics[scale=0.35]{pics/radiator.jpg}
	\caption{\label{fig:radiator}Radiator chłodzący przekaźnik SSR (zdjęcie: Piotr Banaszkiewicz).}
\end{figure}

Elementy te zgodne są z zaleceniami polskiego producenta przekaźników, firmy Relpol \cite{RELPOL}.

\subsection{Chłodzenie przekaźnika półprzewodnikowego}
\label{subsec:chlodzenie_SSR}

Przekaźniki półprzewodnikowe z powodu pewnej niskiej oporności własnej w momencie podłączonego wyjścia mocy wydzielają dużą ilość energii w postaci ciepła.

W ramach pracy inżynierskiej przeprowadzono eksperyment polegający na zmierzeniu temperatury obudowy działającego przekaźnika podczas włączonego obciążenia.

Przekaźnik został umieszczony na radiatorze i pracował przez kilka minut (do momentu ustalenia się temperatury) pod stałym obciążeniem. Kamera termowizyjna ostawiona była tak, by mierzyć temperaturę radiatora oraz obudowy przekaźnika. Tabela \ref{tab:eksperyment} zawiera zestawienie długości czasu działania przekaźnika pod obciążeniem, a także uzyskanej na końcu eksperymentu temperatury, natomiast zdjęcia 5 i 6 ukazują obrazy z kamery termowizyjnej.  %% Linki do tabeli oraz zdjęć

\begin{table}[h]
    \centering
    
    \begin{threeparttable}
        \caption{Parametry eksperymentu obciążenia przekaźnika półprzewodnikowego}
        \label{tab:eksperyment}
        
        \begin{tabularx}{0.9\textwidth}{cccXX}
            \toprule
            Czas obciążenia [min] & Napięcie [V] & Natężenie [A] & Temperatura radiatora\tnote{a} [\degree C] & Temperatura obudowy przekaźnika\tnote{a} [\degree C] \\
            \midrule
            8 & 45 & 36 & 25 & ok. 60 \\
            8 & 42.5 & 43 & 25 -- 35 & ok. 90 \\
            \bottomrule
        \end{tabularx}
        
        \begin{tablenotes}
            \footnotesize
            \item[a] pod koniec pomiaru
        \end{tablenotes}
        
    \end{threeparttable}
\end{table}

\begin{figure}[H]
	\centering
	\includegraphics[scale=0.8]{flir/IR_5167.jpg}
	\caption{\label{fig:flir1}Obraz uzyskany za pomocą kamery termowizyjnej na koniec trwania eksperymentu z mniejszym obciążeniem (zdjęcie: Marek Długosz).}
\end{figure}

\begin{figure}[H]
	\centering
	\includegraphics[scale=0.8]{flir/IR_5170.jpg}
	\caption{\label{fig:flir1}Obraz uzyskany za pomocą kamery termowizyjnej na koniec trwania eksperymentu z większym obciążeniem (zdjęcie: Marek Długosz).}
\end{figure}

Na powyższych zdjęciach najcieplejszym elementem była śruba służąca do przykręcenia kabla zasilającego. Temperatura ciepłego miejsca na obudowie przekaźnika była o około $ 15^{\circ}\mathrm{C} $ niższa; uwzględnione to zostało w tabeli \ref{tab:eksperyment} poprzez odjęcie ok. $ 10^{\circ}\mathrm{C} $ od maksymalnej mierzonej temperatury.

Ponadto należy zauważyć jak niewiele nagrzewał się radiator, do którego przymocowany został przekaźnik. To udowadnia, że jest on wystarczający dla zastosowania w samochodzie.

Na samym końcu należy zauważyć, że przekaźnik nie powinien być w realnych zastosowaniach narażony na tak duże obciążenia przez tak długi czas; według obserwacji, pobór maksymalny chwilowy w trakcie przyspieszania pojazdu i skręcania jednocześnie wynosił około $ 50A $.

%---------------------------------------------------------------------------

\section{Opis części programowej}
\label{sec:opis_cz_programowej}

W tym rozdziale opisano m.in. logikę, algorytmy i struktury danych wykorzystywane przez programy modułów A i B.

\subsection{Schematy cykli-działań}
\label{subsec:schematy_cykli_dzialan}

Poniższe schematy reprezentują logikę, zdarzenia oraz czynności, jakimi posługują się moduły A~i~B.

Dokładny opis pakietów danych oraz momenty, w których są przesyłane, znajduje się w rozdziale \ref{subsec:format_pakietow}.

\raggedbottom

\begin{figure}[H]
	\centering
	\includegraphics[scale=0.8]{schemas/cycle_action_A.png}
	\caption{\label{fig:cycle_action_A}Graf cykli-działań dla modułu A (opracowanie własne).}
\end{figure}

\begin{figure}[H]
	\centering
	\includegraphics[scale=0.8]{schemas/cycle_action_B.png}
	\caption{\label{fig:cycle_action_B}Graf cykli-działań dla modułu B (opracowanie własne).}
\end{figure}

\subsection{Format pakietów}
\label{subsec:format_pakietow}

Każdy pakiet ma długość 4 bajtów:

\begin{equation}
0 \mathrm{x} \underbrace{\mathrm{FC}}_\text{e}  \underbrace{\mathrm{ABCD}}_\text{c} \underbrace{00}_\text{s}
\end{equation}
gdzie
\begin{eqwhere}[2cm]
	\item[$e$] bajt \textit{emergency}. Jeżeli jego wartość to $0xFF$ to moduł B ma natychmiast przejść do trybu awaryjnego. W praktyce oznacza to, że operator wcisnął przycisk ,,STOP'' modułu A.
	\item[$c$] 2 bajty zapisujące wartość liczbową licznika modułu A (typ bez znaku, 16 bitów),
	\item[$s$] bajt stopu, jego wartość zawsze ma być równa $0x00$.
\end{eqwhere}

Istnieją 2 pakiety specjalne przesyłane między modułami A i B w celu synchronizacji i rozpoczęcia działania:

\begin{itemize}
\item pakiet danych aktywacyjnych wysyłany przez moduł A: $0\mathrm{xFFFFFF}00$,
\item pakiet odpowiedzi na dane aktywacyjne wysyłany przez moduł B: $0\mathrm{x11FFFF}00$,
\item ponadto każdy pakiet danych wysyłany przez moduł A już po odebraniu odpowiedzi na dane aktywacyjne zawiera bajt \textit{emergency} ustawiony na wartość $0\mathrm{x}22$ -- za wyjątkiem sytuacji, w której moduł A żąda od modułu B przejścia w tryb awaryjny; wtedy ten bajt ustawiony jest na wartość $0\mathrm{xFF}$.
\end{itemize}

Przykładowe pakiety:

\begin{itemize}
\item $0\mathrm{x}223\mathrm{F}0\mathrm{E}00$: moduł A wysyła informację o braku konieczności przejścia w tryb awaryjny oraz o wartości licznika (16142),
\item $0\mathrm{xFFA}37800$: moduł A wysyła informację o przejściu w tryb awaryjny oraz o ostatniej wartości licznika (41848).
\end{itemize}

\subsection{Maszyna stanów}
\label{subsec:maszyna_stanow}

Oprogramowanie modułów A i B wykorzystuje proste maszyny stanów \cite{Bro69}.

Stany modułu A:

\begin{enumerate}
\item CONNECTING: wysyłanie pakietu danych aktywacyjnych co 1 sekundę, oczekiwanie na odpowiedź,
\item PING: wysyłanie obecnej wartości licznika co 1 sekundę, zwiększanie wartości licznika,
\item STOP: wysyłanie ostatniej wartości licznika oraz ustawionego bajtu \textit{emergency} co $0.5$ sekundy.
\end{enumerate}

Stany modułu B:

\begin{enumerate}
\item WAIT: oczekiwanie na przyjście pakietu danych aktywacyjnych; odesłanie na niego odpowiedzi,
\item RUN: zwiększanie wewnętrznego licznika, odbieranie wartości licznika modułu A, porównywanie obu liczników, sprawdzanie bajtu \textit{emergency},
\item STOP: podtrzymywanie wyłączenia zasilania.
\end{enumerate}

Warunki przejścia między stanami modułu A:

\begin{itemize}
\item CONNECTING $\rightarrow$ PING: gdy pakiet odpowiedzi na dane aktywacyjne wysłany przez moduł B to $0\mathrm{x}11\mathrm{FFFF}00$,
\item PING $\rightarrow$ STOP: gdy operator wciśnie przycisk ,,RESET''.
\end{itemize}

Warunki przejścia między stanami modułu B:

\begin{itemize}
\item WAIT $\rightarrow$ RUN: gdy pakiet danych aktywujących wysłany przez moduł A to $0\mathrm{xFFFFFF}00$,
\item RUN $\rightarrow$ STOP: gdy jeden z warunków jest spełniony:
    \begin{itemize}
    \item bajt \textit{emergency} w przychodzącym pakiecie ma wartość $0\mathrm{xFF}$,
    \item licznik wewnętrzny ma wartość różną od wartości licznika wysłanej przez moduł A w przychodzącym pakiecie,
    \item przez 3 sekundy nie nadeszły żadne dane od modułu A.
    \end{itemize}
\end{itemize}

\subsection{Panel operatorski}
\label{subsec:panel_operatorski}

Obydwa moduły systemu ESAV zawierają minimalny interfejs dla użytkownika. Składa się on odpowiednio z:

\begin{itemize}
\item dla modułu A:
    \begin{itemize}
    \item przycisk ,,STOP'' wywołujący przerwanie wewnętrzne platformy Arduino i w konsekwencji wysłanie polecenia przejścia do trybu awaryjnego,
    \item przycisk ,,RESET'' do łatwiejszego restartu urządzenia,
    \item zieloną diodę informującą o wysyłaniu danych przez port szeregowy,
    \item żółtą diodę włączającą się w momencie naciśnięcia przycisku ,,STOP'' (spełnia ona funkcję wizualnego sprzężenia zwrotnego dla użytkownika),
    \end{itemize}
    
\item dla modułu B:
    \begin{itemize}
    \item przycisk ,,RESET'' do łatwiejszego restartu urządzenia,
    \item zieloną diodę informującą o poprawnym trybie pracy (stan ,,RUN''),
    \item czerwoną diodę informującą o oczekiwaniu na dane wejściowe (szybka pulsacja) lub o przejściu w tryb awaryjny (ciągłe świecenie).
    \end{itemize}
\end{itemize}

\subsection{Szyfrowanie komunikacji}
\label{subsec:szyfrowanie_komunikacji}

Komunikacja pomiędzy modułami (za wyjątkiem przekazywanych danych aktywacyjnych i odpowiedzi na nie) jest w prosty sposób szyfrowana.

Do szyfrowania wykorzystywane są dwie tablice, każda o długości 256, co odpowiada ilości wartości pojedynczego bajtu. Szyfrowanie pojedynczego bajtu polega na odszukaniu w tablicy szyfrującej odpowiadającej mu wartości zaszyfrowanej, a odszyfrowanie polega na odszukaniu w tablicy deszyfrującej wartości oryginalnej odpowiadającej wartości zaszyfrowanej.

Poniższy przykład ilustruje szyfrowanie i deszyfrowanie na przykładzie czterech wartości:

\begin{lstlisting}[frame=single,caption=Pseudokod zbliżony do C pokazujący zasadę szyfrowania i deszyfrowania komunikacji systemu ESAV.]
byte cipher_array[] = {3, 1, 0, 2};
byte decipher_array[] = {2, 1, 3, 0};
byte original = 2;
byte encrypted = cipher_array[original];
assert(original == decipher_array[encrypted]);
\end{lstlisting}

Dzięki wykorzystaniu tej prostej metody nie nastąpił żaden narzut w wielkości przesyłanych pakietów, niestety skutkiem słabszego bezpieczeństwa niż w przypadku stosowania np. algorytmu AES \cite{AESComment}.

\subsection{Programowanie Arduino}
\label{subsec:programowanie_arduino}

Platforma Arduino, na której oparty jest system ESAV, programowana jest w środowisku Arduino IDE (zob. \ref{fig:arduino_ide}) za pomocą języka bardzo zbliżonego do C. Opis jego składni, udostępnionych funkcji i klas dostępny jest w \cite{ArduinoRef}.

Podczas pracy nad projektem stworzono dwa pliki źródłowe, odpowiednio dla modułu A oraz B, a także jedną bibliotekę w C++ współdzieloną przez oba projekty, której zadaniem jest obsługiwanie szyfrowania i deszyfrowania komunikacji pomiędzy modułami.

Pliki źródłowe dostępne są w serwisie GitHub \cite{CodeSource}.

\begin{figure}[H]
	\centering
	\includegraphics[scale=0.5]{pics/arduino_ide.png}
	\caption{\label{fig:arduino_ide}Interfejs programu Arduino IDE używany do programowania na platformę Arduino (zdjęcie: Piotr Banaszkiewicz).}
\end{figure}

\subsection{Kabel do programowania Arduino}
\label{subsec:kabel_do_programowania_arduino}

Platforma Arduino Pro, w przeciwieństwie do innych większych mikrokontrolerów Arduino, nie zawiera programatora na USB. Programowanie Arduino Pro odbywa się poprzez port szeregowy (6 pinów usytuowanych równolegle do płytki PCB, widocznych po lewej stronie na zdjęciu \ref{fig:arduino_pro}).

\begin{figure}[h]
	\centering
	\includegraphics[scale=0.5]{pics/konwerter-usb-uart-ft232-33-5v.jpg}
	\caption{\label{fig:arduino_ide}Opis wyjść używanego konwertera USB-UART (zdjęcie: Botland \cite{KonwerterBotland}).}
\end{figure}

Niestety używany konwerter USB-UART posiada zgodne wyprowadzenia (zob. powyższe zdjęcie), ale w nieodpowiedniej kolejności, co zostało przedstawione w tabeli \ref{tab:piny}.

\begin{table}[h]
    \centering
    
    \begin{threeparttable}
        \caption{Złącza komunikacji szeregowej i zasilania platformy Arduino Pro oraz używanego konwertera USB-UART.}
        \label{tab:piny}
        
        \begin{tabular}{cc}
            \toprule
            Wyprowadzenie Arduino Pro & Wyprowadzenie konwertera USB-UART \\
            \midrule
            RST & GND \\
            TX & TXD \\
            RX & RXD \\
            5V & RTS \\
            GND & CTS \\
            GND & PWR \\
            \bottomrule
        \end{tabular}
        
    \end{threeparttable}
\end{table}

Podłączenia wymagane do prawidłowego zaprogramowania Arduino Pro:

\begin{itemize}
\item RST $\rightarrow$ RTS
\item TX $\rightarrow$ RXD
\item RX $\rightarrow$ TXD
\item 5V $\rightarrow$ PWR
\item GND $\rightarrow$ GND
\end{itemize}

W związku z tym konieczne stało się stworzenie kabla pozwalającego łatwo programować moduły systemu ESAV. Został on zbudowany na podstawie tzw. skrętki ethernetowej (zdjęcie \ref{fig:cable}).

\begin{figure}[h]
	\centering
	\includegraphics[scale=0.3]{pics/cable.jpg}
	\caption{\label{fig:cable}Zdjęcie kabla wykorzystywanego do programowania Arduino (zdjęcie: Piotr Banaszkiewicz).}
\end{figure}

Niestety tak krótki kabel ethernetowy (ok. $20\mathrm{cm}$) ma bardzo dużą sztywność i potrafi podnieść platformę Arduino, ponieważ ma ona niewielką masę.

%---------------------------------------------------------------------------

\section{Zakres zrealizowanych wymagań}
\label{sec:zrealizowane_wymagania}

W poniższej tabeli zestawiono założenia projektu wyszczególnione w rozdziale wraz z informacją o ich spełnieniu (bądź wytłumaczeniem dlaczego nie zostały spełnione) oraz odnośnikiem do rozdziału zawierającego bardziej szczegółowe informacje na temat danej funkcjonalności.

\begin{longtable}{p{.15\textwidth} p{.15\textwidth} p{.15\textwidth} p{.35\textwidth} p{.05\textwidth}}
\caption{Macierz śledzenia wymagań (ang. \textit{requirements traceability matrix} \cite{Hug10}).} \label{tab:spelnione_wymagania} \\

\toprule
Numer wymagania & Wymagane? & Spełnione? & Jak? & Ref. \\
\midrule

1.2.1 & Tak & Tak & Przekaźnik typu \textit{Normally Open}. & \ref{subsec:odlaczanie_zalaczanie_silnika} \\ \hline
1.2.2 & Tak & Tak & Oprogramowanie modułów A~i~B. & \ref{subsec:schematy_cykli_dzialan} \\ \hline
1.2.3 & Tak & Tak & Przekaźnik typu \textit{Normally Open} oraz oprogramowanie modułu B. & \ref{subsec:maszyna_stanow} \\ \hline
1.2.4 & Tak & Tak & Oprogramowanie modułu B. & \ref{subsec:maszyna_stanow} \\ \hline
1.2.5 & Tak & Tak & Oprogramowanie modułu B. & \ref{subsec:maszyna_stanow} \\ \hline
1.2.6 & Tak & Tak & Oprogramowanie modułów A i B. & \ref{subsec:maszyna_stanow} \\ \hline
1.2.7 & Tak & Tak & Podpięcie przycisku ,,RESET'' do pinu RESET modułu B. & \ref{subsec:panel_operatorski} \\ \hline
1.2.8 & Tak & Tak & Przekaźnik typu \textit{Normally Open}. & \ref{subsec:odlaczanie_zalaczanie_silnika} \\ \hline
1.2.9 & Nie & Nie & Zakupione powerbanki umożliwiają ładowanie podczas używania ich, jednakże to w gestii użytkownika jest korzystanie z tej możliwości. & \ref{subsec:zasilanie} \\ \hline
1.3.1 & Tak & Tak & Powerbanki powinny teoretycznie zapewnić zasilanie przez około 50–100 godzin. & \ref{subsec:zasilanie} \\ \hline
1.3.2 & Tak & Tak & Prędkość przesyłania danych między modułami HC-12 wynosi 5000bps. & \ref{subsec:komunikacja} \\ \hline
1.3.3 & Tak & Tak & Przetestowano &  \\ \hline  %% podaj wartosc
1.3.4 & Tak & Tak & Przetestowano &  \\ \hline  %% podaj wartosc
1.4.1 & Tak & Tak & Moduł B odłącza zasilanie modułu, który bezpośrednio zasila silnik samochodu. & \ref{subsec:odlaczanie_zalaczanie_silnika} \\ \hline
1.4.2 & Nie & Nie & Zakupione powerbanki umożliwiają ładowanie podczas używania ich, jednakże to w gestii użytkownika jest korzystanie z tej możliwości. & \ref{subsec:zasilanie} \\ \hline
1.5.1 & Tak & Tak & Wejście sterowania przekaźnika jest zabezpieczone diodą gaszącą. & \ref{subsec:odlaczanie_zalaczanie_silnika} \\ \hline  %% uaktualnij!
1.5.2 & Tak & Tak & Komunikacja pomiędzy modułami polega na wymianie szyfrowanego licznika. & \ref{subsec:szyfrowanie_komunikacji} \\ \hline
1.6.1 & Nie & Nie & Poziom naładowania jest wyświetlany na obudowie powerbanku, więc nie ma konieczności dodatkowego monitorowania przez moduł B. & \ref{subsec:zasilanie} \\ \hline
1.6.2 & Tak & Tak & Spełnione dosłownie. & \ref{subsec:panel_operatorski} \\ \hline
1.6.3 & Tak & Tak & Spełnione dosłownie. & \ref{subsec:panel_operatorski} \\ \hline
1.6.4 & Tak & Tak & Spełnione dosłownie. & \ref{subsec:panel_operatorski} \\ \hline
1.6.5 & Nie & Nie & Poziom naładowania jest wyświetlany na obudowie powerbanku, więc nie ma konieczności dodatkowego monitorowania przez moduł B. & \ref{subsec:zasilanie} \\ \hline
1.6.6 & Tak & Tak & Spełnione dosłownie. & \ref{subsec:panel_operatorski} \\ \hline
1.6.7 & Tak & Tak & Spełnione dosłownie. & \ref{subsec:obudowy_montaz} \\ \hline
1.6.8 & Tak & Tak & Wysłanie pakietu danych jest rozumiane jako ,,PING''. & \ref{subsec:format_pakietow} \\ \hline
1.6.9 & Tak & Tak & Opisane w formacie pakietów. & \ref{subsec:format_pakietow} \\ \hline
1.6.10 & Nie & Nie & Niewymagane. &  \\ \hline

\bottomrule
\end{longtable}

%---------------------------------------------------------------------------

\section{Podsumowanie}
\label{sec:podsumowanie_specyfikacja_projektu}


